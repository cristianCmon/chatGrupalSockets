% [Tamaño principal de la fuente del documento, tamaño del papel, con título (crea salto de página)]{Tipo de documento}
\documentclass[12pt, a4paper, titlepage]{article}
\usepackage[utf8]{inputenc}
% Traduce expresiones al español
\usepackage[spanish]{babel}
\usepackage{setspace}

\usepackage{graphicx}
\usepackage{geometry}
\geometry{margin=2.5cm}
\graphicspath{ {./imagenes/} }

% Permite utilizar labeling para listar de forma personalizada
\usepackage{scrextend}
% Permite centrar verticalmente m{}
\usepackage{array}

\title{\LARGE \textbf{Documentación de Carrera de coches con Synchronized} \\[2ex] \Large Programación de Servicios y Procesos}
\author{\\[20ex]Cristian Fernández}
\date{\today}

\begin{document}
\maketitle

\doublespacing
\tableofcontents % Índice
\newpage

\section{Descripción del problema}
\noindent \\El docente encargado de impartir el módulo de \textit{Programación de Servicios y Procesos}, Don Roberto Castro Liste,
ha encargado la realización de una tarea que consiste en crear un programa en \textbf{JavaFX} que simule una carrera de coches,
creando una clase que represente a los participantes de la misma, y utilizar un método \textbf{Synchronized} para evitar posibles \textit{Race Conditions} entre hilos. \\ \\
El programa debe seguir estas pautas:\\
\begin{itemize}
    \item Diseñar una interfaz que exprese de forma clara el funcionamiento de los hilos.
    \item Crear una clase ParticipanteCarrera que contenga al menos una propiedad Rectangulo, que representará visualmente el coche.
    \item Crear varios objetos participante en la clase principal e iniciarlos como hilos independientes.
    \item Simular el desplazamiento de los vehículos y su correcto paso por meta en un método synchronized.
    \item El programa debe continuar su hasta que todos los coches crucen la línea de meta y se observen las posiciones de llegada de los participantes.
\end{itemize}
\newpage

\section{Requisitos funcionales}
\noindent \\A continuación se listan los requisitos funcionales del sistema:\\
%\begin{description} sería equivalente pero items en negrita por defecto
\begin{labeling}{ RF-5:} % el segundo parámetro marca el espaciado entre ítem y texto
    \item[ \textbf{ RF-1}:] Ver una aviso que informa de una mejora de experiencia si hay sonido.
    \item[ \textbf{ RF-2}:] Pulsar un botón que inicie la ejecución del programa.
    \item[ \textbf{ RF-3}:] Mostrar dinámicamente todo el proceso de ejecución de la carrera.
\end{labeling}
\newpage

\section{Requisitos no funcionales}
\noindent \\A continuación se listan los requisitos NO funcionales del sistema:\\
\begin{labeling}{ RNF-5:}
    \item[ \textbf{RNF-1}:] El tiempo de respuesta al pulsar el botón Iniciar no debe superar los 2 segundos.
    \item[ \textbf{RNF-2}:] La intefaz gráfica de usuario ha de ser fácilmente entendible.
    \item[ \textbf{RNF-3}:] El sonido debe reproducirse de forma fluida y sin cortes.
    \item[ \textbf{RNF-4}:] Cada acción realizada por los hilos debe ser lo suficientemente descriptiva para el usuario.
    \item[ \textbf{RNF-5}:] La interfaz debe avisar de forma clara que la ejecución de la aplicación ha terminado.
\end{labeling}
\newpage

\section{Casos de uso}
\noindent \\A continuación se muestran los casos de uso mediante la siguiente tabla:\\ \\ \\
\begin{tabular}{| m{5cm} | m{8cm} |} % | pinta las columnas verticales
\hline % pinta las filas horizontales
\textbf{Caso de uso} & \textbf{Descripción} \\
\hline
Leer recomendación & El usuario entiende que la experiencia con la aplicación mejora con sonido activado. \\
\hline
Iniciar aplicación & El usuario puede iniciar la aplicación pulsando un botón. \\
\hline
\end{tabular}
\newpage

\section{Historias de usuario}
\noindent \\A continuación se muestran las historias de usuario mediante la siguiente tabla:\\ \\ \\
\begin{tabular}{| m{2cm} | m{2cm} | m{4cm} | m{4cm} |}
\hline
\textbf{ID} & \textbf{Como...} & \textbf{Quiero...} & \textbf{Para...} \\
\hline
HU-01 & Usuario & Pulsar un botón & Iniciar la aplicación \\
\hline
HU-02 & Usuario & Contemplar una carrera de coches & Divertirme y entender la sincronía \\
\hline
\end{tabular}
\newpage

\section{Herramientas de desarrollo}
\noindent \\\textbf{IntelliJ Community Edition} es un IDE gratuito y de código abierto, ideal para el desarrollo en lenguajes JVM y Android.
Sus características principales incluyen un potente editor de código con autocompletado inteligente y refactorización, soporte para control de versiones,
un depurador integrado, y herramientas de pruebas unitarias. Está disponible para todas las plataformas.  \\
\\Características principales: \\
\begin{itemize}
    \item \textbf{Editor de código potente:} Ofrece resaltado de sintaxis, análisis en tiempo real, sugerencias de código, e inspecciones y correcciones rápidas.
    \item \textbf{Soporte para múltiples lenguajes:} Es compatible con Java, Kotlin, Groovy, Scala, y otros lenguajes JVM.
    \item \textbf{Integración con control de versiones:} Permite la integración con sistemas como Git, SVN, y Mercurial.
\end{itemize}
\newpage

\section{Lenguajes de programación}
\noindent \\\textbf{JavaFX} se caracteriza por su capacidad de crear aplicaciones enriquecidas visualmente, integrando multimedia (audio, video) y gráficos vectoriales.
Permite desarrollar aplicaciones multiplataforma (escritorio, web, móvil, smart TV). Otras características clave son
la integración fluida con el ecosistema Java y herramientas como FXML y Scene Builder para un desarrollo más eficiente y colaborativo. \\
\\Características principales: \\
\begin{itemize}
    \item \textbf{Interfaces gráficas de usuario (GUI) enriquecidas:} Permite crear interfaces modernas y atractivas con animaciones, gráficos vectoriales, y elementos personalizables.
    \item \textbf{Integración multimedia:} Facilita la inclusión de contenido de audio, video y otros medios dentro de las aplicaciones.
    \item \textbf{Desarrollo declarativo (FXML):} Utiliza FXML, un lenguaje basado en XML, para describir la interfaz de usuario, lo que facilita el trabajo colaborativo entre diseñadores y desarrolladores.
\end{itemize}
\newpage

\section{Interfaz de la aplicación}
\noindent \\A continuación unas capturas de las diferentes vistas de la interfaz.\\
\begin{figure}[hbt!]
    \centering
    \includegraphics[width=0.8\textwidth]{Screenshot_1.png}
    \caption{Vista inicial, muestra advertencia/consejo al usuario}
\end{figure}
\newpage
\begin{figure}[hbt!]
    \centering
    \includegraphics[width=1\textwidth]{Screenshot_2.png}
    \caption{Inicio de ejecución del programa}
\end{figure}
\noindent \\
\begin{figure}[hbt!]
    \centering
    \includegraphics[width=1\textwidth]{Screenshot_3.png}
    \caption{Fin de ejecución del programa}
\end{figure}

\end{document}